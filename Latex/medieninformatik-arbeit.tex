\documentclass[11pt,a4paper,twoside]{article}

% LaTeX-Umsetzung der "Richtlinien f�r Projekt- und Diplomarbeiten"
% der LFE Medieninformatik, LMU M�nchen. (Autor: Richard Atterer, 27.9.2006, 23.10.2007), Bug-Fixing Mark Kaczkowski (23.6.2008)

\usepackage[T1]{fontenc} % sonst geht \hyphenation nicht mit Umlauten
\usepackage[latin1]{inputenc} % man kann schreiben ����, statt "a"o"u"s
%\usepackage[utf8]{inputenc} % wie oben, aber UTF-8 als Encoding statt ISO-8859-1 (latin1)
\usepackage[ngerman,english]{babel} % deutsche Trennregeln, "Inhaltsverzeichnis" etc.
%\usepackage{ngerman} % Alternative zum Babel-Paket oben
\usepackage{mathptmx} % Times-Roman-Schrift (auch f�r mathematische Formeln)

% Zum Setzen von URLs
\usepackage{color}
\definecolor{darkred}{rgb}{.25,0,0}
\definecolor{darkgreen}{rgb}{0,.2,0}
\definecolor{darkmagenta}{rgb}{.2,0,.2}
\definecolor{darkcyan}{rgb}{0,.15,.15}
\usepackage[plainpages=false,bookmarks=true,bookmarksopen=true,colorlinks=false,
  linkcolor=darkred,citecolor=darkgreen,filecolor=darkmagenta,
  menucolor=darkred,urlcolor=darkcyan]{hyperref}

% pdflatex: Bilder in den Formaten .jpeg, .png und .pdf
% latex: Bilder im .eps-Format
\usepackage{graphicx}

\usepackage{fancyhdr} % Positionierung der Seitenzahlen
\fancyhead[LE,RO,LO,RE]{}
\fancyfoot[CE,CO,RE,LO]{}
\fancyfoot[LE,RO]{\Roman{page}}
\renewcommand{\headrulewidth}{0pt}
\setlength{\headheight}{13.6pt} % behebt headheight Warning

% Korrektes Format f�r Nummerierung von Abbildungen (figure) und
% Tabellen (table): <Kapitelnummer>.<Abbildungsnummer>
\makeatletter
\@addtoreset{figure}{section}
\renewcommand{\thefigure}{\thesection.\arabic{figure}}
\@addtoreset{table}{section}
\renewcommand{\thetable}{\thesection.\arabic{table}}
\makeatother

\sloppy % Damit LaTeX nicht so viel �ber "overfull hbox" u.�. meckert

% R�nder
\addtolength{\topmargin}{-16mm}
\setlength{\oddsidemargin}{25mm}
\setlength{\evensidemargin}{35mm}
\addtolength{\oddsidemargin}{-1in}
\addtolength{\evensidemargin}{-1in}
\setlength{\textwidth}{15cm}
\addtolength{\textheight}{34mm}
%______________________________________________________________________

\begin{document}

\pagestyle{empty} % Vorerst keine Seitenzahlen
\pagenumbering{alph} % Unsichtbare alphabetische Nummerierung

\begin{center}
\textsc{Ludwig-Maximilians-Universit�t M�nchen}\\
Department ``Institut f�r Informatik''\\
Lehr- und Forschungseinheit Medieninformatik\\
Prof.\ Dr.\ Heinrich Hu�mann

\vspace{5cm}
{\large\textbf{Masterarbeit}}\vspace{.5cm}

{\LARGE The Use of Pursuits in Virtual Reality}\vspace{1cm}

{\large Carl Oechsner}\\\href{mailto:oechsner.carl@gmail.com}{oechsner.carl@gmail.com}

\end{center}
\vfill

\begin{tabular}{ll}
Bearbeitungszeitraum: & 1. 11. 2016 bis 1. 6. 2017\\
Betreuer: & Mohamed Khamis\\
%Externer Betreuer: & Manfred Manager\\
Verantw. Hochschullehrer: & Prof. Butz ODER Prof. Hu�mann
\end{tabular}
%______________________________________________________________________

\clearpage
\section*{Zusammenfassung}

Kurzzusammenfassung der Arbeit, maximal 250 W�rter.

\selectlanguage{english}
\section*{Abstract}

Short abstract of the work, maximum of 250 words.

\selectlanguage{ngerman}
\clearpage
\section*{Task}

Kopie der Original-Aufgabenstellung

\section*{Motivation}

Keine.

\vfill % Sorgt daf�r, dass das Folgende an das Seitenende rutscht

\noindent Ich erkl�re hiermit, dass ich die vorliegende Arbeit
selbstst�ndig angefertigt, alle Zitate als solche kenntlich gemacht
sowie alle benutzten Quellen und Hilfsmittel angegeben habe.

\bigskip\noindent M�nchen, \today

\vspace{4ex}\noindent\makebox[7cm]{\dotfill}

%______________________________________________________________________

\cleardoublepage
\pagestyle{fancy}
\pagenumbering{roman} % R�mische Seitenzahlen
\setcounter{page}{1}

% Inhaltsverzeichnis erzeugen
\tableofcontents

%Abbildungsverzeichnis erzeugen - normalerweise nicht n�tig
%\cleardoublepage
%\listoffigures
%______________________________________________________________________

\cleardoublepage

% Arabische Seitenzahlen
\pagenumbering{arabic}
\setcounter{page}{1}
% Ge�ndertes Format f�r Seitenr�nder, arabische Seitenzahlen
\fancyhead[LE,RO]{\rightmark}
\fancyhead[LO,RE]{\leftmark}
\fancyfoot[LE,RO]{\thepage}

\section{Introduction}

Oh Mann, noch \emph{so} viele Seiten zu f�llen...\\
...und wieso muss bei diesem Format so viel auf eine Seite passen!?
%______________________________________________________________________

% Der Befehl \cleardoublepage erscheint nur vor \section, nicht vor
% den "kleineren" Gliederungsbefehlen wie \subsection!
\cleardoublepage % Neue rechte Seite anfangen
\section{Conceptual Framework}
\subsection{Eye Tracking}
\subsection{Virtual Reality}
\subsection{Pursuits}

\cleardoublepage
\section{Design Space}
\subsection{User Bound Objects}
\subsection{World Bound Objects}

\cleardoublepage
\section{Implementation}
\label{Implementation}

\cleardoublepage
\section{Evaluation}
In order to evaluate the system described in \ref{Implementation} we decided to focus on the user-bound setting. INSERT GOOD REASON HERE

\subsection{Trial Study}
\subsubsection{Procedure}

In order to exclude certain settings and to identify limits within which the main study should take place, a qualitative trial study was conducted with 14 participants. The virtual objects were decided to be numbered cubes to easily tell the participants where to fixate on. The available visible space was determined empirically and reached from -3 to 3 units both horizontally and vertically. All objects had a distance of 6.78 units to the camera and were highlighted with a halo effect when a fixation was detected. There were four variables tested:

\begin{enumerate}
	\item \textbf{Number}: The basic object was a cube with an edge length of 0.7 units moving on a horizontal trajectory that reached from one edge of the visible space to the other and back thus having a length of 6 units. To determine up to which number the system can correctly determine which object is selected, \{3,5,7,9\} of these objects were shown at the same time starting their movement at a random position on their trajectory which had a vertical distance of 0.7 units. As the Pursuits approach would not work if all objects had the same velocity, they were given different speeds with a difference of at least 0.2 units per second.
	Additionaly, circular trajectories were investigated. \{2,3,5\} numbered cubes were moving clockwise along a radial path with a radius of 2 units with a velocity of 45 degrees per second.
	
	\item \textbf{Position}: In this scene there were always three objects shown at a time at the edges of the visible area to determine if the trajectory position has an impact on tracking performance. Three objects on horizontal trajectories were positioned at the top or at the bottom and three objects on vertical trajectories were positioned at the left or at the right edge of the FOV. The trajectories always had a distance of 1 unit to each other and the velocity delta was 0.4 units per second.
	
	\item \textbf{Size}: Cubes of three different sizes were compared to each other. Again there were always shown three cubes on horizontal trajectories with an edge length of \{0.3,1.5\} units at a time to compare them to the standard 0.7 unit cubes. Their trajectories hat a distance of one unit in the small and 2.5 units in the large case to prevent overlapping. Their velocities differed by one unit per second.
	
	\item \textbf{Trajectory Shape}: The three different trajectory shapes \{horizontal, vertical, circular\} are compared to each other in one scene each being equipped by a 0.7 unit cube. The linear trajectories had a length of 6 units and a speed of 2 units per second, the circular trajectory had a radius of 2 units and a speed of 45 degrees per second. 
\end{enumerate}

\label{correlator}
During the course of the trial study, most reliable results were achieved with a threshold of 0.4, a correlation frequency of 3.33, a correlation time window of 300ms and a correlation averaging window of 900 milliseconds meaning that the average of the three last coefficients for an object was compared to the threshold.

After each setting the users were asked which characteristics they found most convenient and reliable to use. Based on this evaluation we came to the following results.
\subsubsection{Results and Discussion}

All participants but one preferred circular over linear trajectories, because they were more convenient to follow. One participant mentioned that this was due to the larger distance between the objects. Furthermore, all but one person preferred smaller cubes over large cubes. Most of them agreed on them being easier to follow and to focus on. When it came to the objects' positions many agreed that differences between top and bottom (respectively left and right) were hard to tell. When they had to decide which position was more convenient, there was no discernible tendency except least users voted for the upper position. Regarding the number of objects, most experimentees preferred no more than five objects. As for the linear trajectories this was to the perceived drop in detection performance. Because there were hardly any detection errors with five objects on a circular trajectory three more cubes were added to it moving counter-clockwise to sound out the limits of this setting. Although detection was still fairly reliable most agreed on a limit of five objects due to visual clutter.
As there was visual feedback on the selection provided the participants knew if a selection was right or wrong. If the software detected fixations in one setting better than in another this might have had an influence on their opinion. 
% cubes, numbers, find flaws in the framework, optimize logging for a straightforward analysis process, object size and trajectory size should be considered separately and have a certain ratio, users prefer a limited area they can fixate their eyes on

\subsection{Main Study}

The main study had two primary goals: First, to find design guidelines for traceable objects in a virtual reality setting employing Pursuits, i.e. limits and ratios between object and trajectory size that improve performance. Second, by adding a task in which the user is moving within the VR area, we wanted to figure out to which amount pursuing objects distracts from the main task. The Correlator settings found credible in \ref{correlator} were used throughout the main study.

\subsubsection{Abstract Scenario}

In the abstract scenario we aimed for investigating certain variables in an isolated manner to acquire quantitative data for further analysis. There should be no biasing clues or embellishments distracting participants from their task and to make sure the observed effects are solely due to manipulations of the independent variables. Another key difference to the Use Case Scenario is that no feedback is provided on which object is detected which might have an impact on subjective ratings.

\begin{description}
	\item [Independent variables]: Trajectory radius \{0.5, 1.5, 2.5\}, Object size \{0.3, 0.65, 0.9\}
	\item [Dependent variables]: Selection time, error rate
	\item [Fixed variables]: Object velocity \{45 �/s\}, position \{center\}, trajectory shape \{circular\}
\end{description}

As neither \textit{object velocity} and \textit{position} had a mentionable effect and circular \textit{trajectories} were rated both most reliable and comfortable to use in the trial study, these variables were decided to be fixed throughout the main study. Furthermore, as detection was rated most agreeable with up to five objects, this was the final choice for the number of objects per trajectory.

\subsubsection*{Static User}

The seated user is shown every combination of object size and trajectory size in a black surrounding in a randomized Latin square design. There are five coloured cubes of the same size moving along one circular trajectory. One of the cubes is coloured red, the other four are coloured blue. At each trial the user is told to fixate on the red cube. A cube counts as selected as soon as its coefficient is greater than the others' and exceeds the threshold. It counts as an error if a blue cube or nothing is selected. The trajectory centre is in the centre of the participant's field of view.

\subsubsection*{Moving User}

In the virtual environment, the task is to walk up to white spots on the floor that appear in random locations. Meanwhile the same task is to be performed as in the static setting. When a spot is reached and the participant is searching for the next location, the current task is muted, and the cubes disappear. As soon as they start walking again, the task is restarted. Time and error rate measurements are performed in the same way as in the static task. For safety reasons the chaperone borders are visible.

\subsubsection{Use Case Scenario}
Besides the Abstract Scenario we wanted to examine the applicability of Pursuits in concrete use cases. Again we separated between a static and a moving scenario and recorded performance data in the form of selection time and error rate. Additionally the participant is provided input feedback and at the end of each setting qualitative feedback is gathered with a NASA-TLX questionnaire.

\subsubsection*{Static User}
The experimentee is asked to walk up to a virtual ATM machine and enter a PIN that is given to them by the experimenter. A number can be entered by looking at a cube tagged with the respective digit. The numbers one to five are moving along one circular trajectory in a clockwise manner, the numbers five to nine and zero are moving on a second trajectory in a counter-clockwise manner. There is a counter for every cube that is increased whenever the software detects a fixation (just like in the abstract setting). After a time limit ${t}_{s}\in \{0.3,0.5,0.7,0.9\}$ TO BE DETERMINED!! the cube with the highest counter is considered as input. Every time an input is accepted, the user gets visual feedback in form of a temporal highlighting of all cubes so it is clear that they can move on to the next number.

\subsubsection*{Moving User}

For the moving user task we decided on a game setting in which the user is asked to ``shoot'' at meteors that move into the field of view. Like in the abstract case the target object is coloured red while the other objects are coloured blue. There is only one target object visible at a time to clearly detect wrong selections. The positions and shapes of the object trajectories are randomly chosen by the software. By also employing linear trajectories this scenario enables a direct comparison to circular trajectories.

\subsection{Results}

\subsection{Discussion}

\cleardoublepage
\section{Conclusion}

\cleardoublepage
\section{Future Work}

%\begin{figure}%[btph]
  %% Datei ``beispielbild.eps'' oder ``beispielbild.png'', zentriert
  %\begin{center}\includegraphics{beispielbild}\end{center}

  %% Datei auf 8cm Breite verkleinert/vergr��ert
  %\includegraphics[width=8cm]{beispielbild}
  %% Datei auf ganze Breite des Texts vergr��ert
  %\includegraphics[width=\columnwidth]{beispielbild}
  %% Datei auf 60% der Textbreite verkleinert/vergr��ert
  %\includegraphics[width=.6\columnwidth]{beispielbild}
  %% Weitere Optionen (Ausschnitt, drehen etc.) in der Doku zum graphicx-Paket

%  \begin{center}\LARGE [BILD]\end{center}
%  \caption{Bildunterschrift}
%  \label{fig:beispielbild}
%\end{figure}


%Siehe Abbildung \ref{fig:beispielbild} oder einschl�gige Literatur, z.B.
%\cite[Seite 6]{Ivory01} oder \cite{NielsenAlertbox}.
%
%\bigskip % Gr��erer Abstand zum vorherigen Absatz
%\textbf{Hinweis:} Die Verweise im generierten PDF (HTTP-Links, Verweise auf Kapitel oder Bilder) sind leicht eingef�rbt. Wer das nicht will, z.B. weil es die Druckkosten erh�ht, kann am Anfang des Dokuments \texttt{linkcolor} usw. auf ``black'' setzen.


%\subsection{Medien}
%
%\begin{figure}
%  \begin{center}\LARGE [BILD]\end{center}
%  \caption{Noch ein Bild}
%  \label{fig:beispielbild2}
%\end{figure}
%
%\begin{itemize}
%  \item Gesellschaftliche Medien
%  \item Technische Medien
%\end{itemize}
%
%
%\subsection{Informatik}
%
%
%\subsection{Medieninformatik}
%
%\begin{description}
%  \item[Medienwirkung:] Ein Spezialfach der Kommunikationswissenschaft. F�r das erfolgreiche Studium des Anwendungsfachs Mediengestaltung ist eine k�nstlerische Begabung erforderlich.
%  \item[Medienwirtschaft:] Ein Spezialfach der Betriebswirtschaftslehre
%  \item[Mediengestaltung:] Ein Spezialfach der Kunstwissenschaft
%\end{description}
%
%\subsubsection{Was Sie schon immer wissen wollten, aber nie zu fragen
%  wagten}
%
%\paragraph{�berschrift}
%Diese �berschrift erscheint fettgedruckt am Anfang des Absatzes. \cite{NielsenAlertbox}
%
%\subsubsection{Was Sie nicht wissen wollten}
%
%Text text textextext\footnote{Oder so �hnlich}.

%\_____________________________________________________________________

\cleardoublepage
%\section{Zusammenfassung}
%
%\begin{figure}
%  \begin{center}\LARGE [BILD]\end{center}
%  \caption{Bild}
%  \label{fig:beispielbild3}
%\end{figure}
%______________________________________________________________________

\cleardoublepage
\fancyhead[LE,RO,LO,RE]{} % Keine Kopfzeile mehr oben auf jeder Seite
\section*{Inhalt der beigelegten CD}
%______________________________________________________________________

\cleardoublepage
\begin{thebibliography}{99}

%\bibitem{Ivory01}
%
%  M.\ Y.\ Ivory, M.\ Hearts:
%  \href{http://www.ischool.washington.edu/myivory/thesis/thesis.pdf}{%
%    An Empirical Foundation for Automated Web Interface Evaluation}.
%  Ph.D. thesis, University of California at Berkeley, 2001


\cleardoublepage
\hspace{-\leftmargin}{\Large\bfseries Web-Referenzen} % W�ster Hack %-|

%\bibitem{NielsenAlertbox}
%
%  J.\ Nielsen: Alertbox: Current Issues in Web Usability
%  \url{http://useit.com/alertbox/}, accessed April~24, 2005.

\end{thebibliography}

\end{document}
